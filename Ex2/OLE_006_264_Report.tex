%	INF219
%	Oskar L. F. Lerivåg
%	INF264 Exersice 2

\documentclass[a4paper, 12pt]{article}

%Bookmarks
\usepackage[colorlinks=true,urlcolor=cyan,linkcolor=black,citecolor=red,bookmarksopen=true]{hyperref}
\usepackage{bookmark}

\usepackage[utf8]{inputenc}
\usepackage{amsmath}
\usepackage{pgf,tikz}
\usepackage{mathrsfs}
\usepackage{listings}
\usetikzlibrary{arrows}
\usepackage{amssymb}
\usepackage{url}
\usepackage{epigraph}

%Subfile
\usepackage{subfiles}

   \usepackage[breakable]{tcolorbox}
    \usepackage{parskip} % Stop auto-indenting (to mimic markdown behaviour)
    
    \usepackage{iftex}
    \ifPDFTeX
    	\usepackage[T1]{fontenc}
    	\usepackage{mathpazo}
    \else
    	\usepackage{fontspec}
    \fi

    % Basic figure setup, for now with no caption control since it's done
    % automatically by Pandoc (which extracts ![](path) syntax from Markdown).
    \usepackage{graphicx}
    % Maintain compatibility with old templates. Remove in nbconvert 6.0
    \let\Oldincludegraphics\includegraphics
    % Ensure that by default, figures have no caption (until we provide a
    % proper Figure object with a Caption API and a way to capture that
    % in the conversion process - todo).
    \usepackage{caption}
    \DeclareCaptionFormat{nocaption}{}
    \captionsetup{format=nocaption,aboveskip=0pt,belowskip=0pt}

    \usepackage[Export]{adjustbox} % Used to constrain images to a maximum size
    \adjustboxset{max size={0.9\linewidth}{0.9\paperheight}}
    \usepackage{float}
    \floatplacement{figure}{H} % forces figures to be placed at the correct location
    \usepackage{xcolor} % Allow colors to be defined
    \usepackage{enumerate} % Needed for markdown enumerations to work
    \usepackage{geometry} % Used to adjust the document margins
    \usepackage{amsmath} % Equations
    \usepackage{amssymb} % Equations
    \usepackage{textcomp} % defines textquotesingle
    % Hack from http://tex.stackexchange.com/a/47451/13684:
    \AtBeginDocument{%
        \def\PYZsq{\textquotesingle}% Upright quotes in Pygmentized code
    }
    \usepackage{upquote} % Upright quotes for verbatim code
    \usepackage{eurosym} % defines \euro
    \usepackage[mathletters]{ucs} % Extended unicode (utf-8) support
    \usepackage{fancyvrb} % verbatim replacement that allows latex
    \usepackage{grffile} % extends the file name processing of package graphics 
                         % to support a larger range
    \makeatletter % fix for grffile with XeLaTeX
    \def\Gread@@xetex#1{%
      \IfFileExists{"\Gin@base".bb}%
      {\Gread@eps{\Gin@base.bb}}%
      {\Gread@@xetex@aux#1}%
    }
    \makeatother

    % The hyperref package gives us a pdf with properly built
    % internal navigation ('pdf bookmarks' for the table of contents,
    % internal cross-reference links, web links for URLs, etc.)
    \usepackage{hyperref}
    % The default LaTeX title has an obnoxious amount of whitespace. By default,
    % titling removes some of it. It also provides customization options.
    \usepackage{titling}
    \usepackage{longtable} % longtable support required by pandoc >1.10
    \usepackage{booktabs}  % table support for pandoc > 1.12.2
    \usepackage[inline]{enumitem} % IRkernel/repr support (it uses the enumerate* environment)
    \usepackage[normalem]{ulem} % ulem is needed to support strikethroughs (\sout)
                                % normalem makes italics be italics, not underlines
    \usepackage{mathrsfs}
    
        \geometry{verbose,tmargin=1in,bmargin=1in,lmargin=1in,rmargin=1in}

%Images%
\usepackage{graphicx}
\usepackage{float}

%Margins
\usepackage{geometry}
\geometry{a4paper, margin=3cm}

%Citations
\usepackage[round]{natbib}
\bibliographystyle{plainnat}

\newcommand{\mysection}[1]{\section*{#1} \addcontentsline{toc}{section}{#1}}
\newcommand{\mysubsection}[1]{\subsection*{#1} \addcontentsline{toc}{subsection}{#1}}
\newcommand{\mysubsubsection}[1]{\subsubsection*{#1} \addcontentsline{toc}{subsubsection}{#1}}

\newcommand{\myFigure}[3]{\begin{figure}[h!]\centering\includegraphics[scale=#1]{figures/#2}\caption{#3}\end{figure}}

\newcommand{\mycitation}[1]{[\citet{#1}]}

\begin{document}

    % % % % % % % % % % % % % % % % %
    %
    %	FRONT PAGE
    %
    \input{./uib_frontpage.tex}

    % % % % % % % % % % % % % % % % %
    %
    %	TABLE OF CONTENTS
    %
    \pdfbookmark{\contentsname}{toc}
    \tableofcontents
    \newpage
    
    \mysection{Foreword}
    There was a large inconsistency between the paper and the code comments. This has resulted in some answers being in the code, while the rest is in the sections here.

    % % % % % % % % % % % % % % % % %
    %
    %	BEGIN
    %
	\mysection{Univariate Linear Regression}
	\mysubsection{Linear regression in closed form}
	
	\begin{minipage}[t]{0.5\textwidth}
	The linear regression does "fit" the points, but won't perfectly fit all. It is though possible to calculate the avg. distance or "inaccuracy".
	\end{minipage} \quad \vline \quad \begin{minipage}[t]{0.5\textwidth}
	\[
		\hat{w} = \begin{bmatrix}
		6.45478709 \\
		5.02129039
		\end{bmatrix}
	\]
	\end{minipage}
	
	\mysubsection{Gradient descent}
	\[	
		\frac{\partial}{\partial w_0}
		\left( \frac{1}{n} \displaystyle\sum_{i=1}^{n} [y_i - (w_1 x + w_0)]^2 \right)
		=
		\frac{\partial}{\partial w_0}
		\left( \frac{1}{n} \displaystyle\sum_{i=1}^{n} [-2(y_i-(w_1x+w_0))] \right)
	\]	

	\[	
		\frac{\partial}{\partial w_1}
		\left( \frac{1}{n} \displaystyle\sum_{i=1}^{n} [y_i - (w_1 x + w_0)]^2 \right)
		=
		\frac{\partial}{\partial w_1}
		\left( \frac{1}{n} \displaystyle\sum_{i=1}^{n} [-2x(y_i-(w_1x+w_0))] \right)
	\]

	While doing linear regression one tries to get as close to the correct answer as possible. this is done by calculating the step-size by the gradient from the derivative and the learning-rate. The learning-rate $learning_rate = 1$ is so large that the gradient skips the point we are trying to reach. Therefore we miss the point by a mile.
	\\\\
	By using a $learning_rate = 0.1$ we get really close to what sklearn has. But it is far from perfect. We could use $learning_rate = 0.3$ or add more steps to get closer. The "correct" approach would though be to continue as long as the gradient is larger than zero or until a max nr. of steps has been used.
	
	\newpage
	
	\mysection{Code}
	\subfile{./LinearModels_ex2.tex}
		
	
    % % % % % % % % % % % % % % % % %
    %
    %	ADD REFERANCES
    %
    %\bibliography{citation-db}
    %\addcontentsline{toc}{section}{References}

\end{document}