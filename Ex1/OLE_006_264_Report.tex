%	INF219
%	Oskar L. F. Lerivåg
%	INF264 Exersice 1

\documentclass[a4paper, 12pt]{article}

%Bookmarks
\usepackage[colorlinks=true,urlcolor=cyan,linkcolor=black,citecolor=red,bookmarksopen=true]{hyperref}
\usepackage{bookmark}

\usepackage[utf8]{inputenc}
\usepackage{amsmath}
\usepackage{pgf,tikz}
\usepackage{mathrsfs}
\usepackage{listings}
\usetikzlibrary{arrows}
\usepackage{amssymb}
\usepackage{url}
\usepackage{epigraph}

%Images%
\usepackage{graphicx}
\usepackage{float}

%Margins
\usepackage{geometry}
\geometry{a4paper, margin=3cm}

%Citations
\usepackage[round]{natbib}
\bibliographystyle{plainnat}

\newcommand{\mysection}[1]{\section*{#1} \addcontentsline{toc}{section}{#1}}
\newcommand{\mysubsection}[1]{\subsection*{#1} \addcontentsline{toc}{subsection}{#1}}
\newcommand{\mysubsubsection}[1]{\subsubsection*{#1} \addcontentsline{toc}{subsubsection}{#1}}

\newcommand{\myFigure}[3]{\begin{figure}[h!]\centering\includegraphics[scale=#1]{figures/#2}\caption{#3}\end{figure}}

\newcommand{\mycitation}[1]{[\citet{#1}]}

\begin{document}

    % % % % % % % % % % % % % % % % %
    %
    %	FRONT PAGE
    %
    \input{./uib_frontpage.tex}

    % % % % % % % % % % % % % % % % %
    %
    %	TABLE OF CONTENTS
    %
    \pdfbookmark{\contentsname}{toc}
    \tableofcontents
    \newpage

    % % % % % % % % % % % % % % % % %
    %
    %	BEGIN
    %
	\mysection{Mathematical warm up}
	\mysubsection{Matrices}
	{
	\begin{enumerate}
	\item
	Following that $ad - bc \neq 0 $ we know that these matrices are invertible as both $1*2-1*-1\neq0$ and $2*-2-1*1\neq0$ holds true.
	\[
		A^{-1} = \begin{bmatrix}
		1 & -1 \\
		1 & 2
		\end{bmatrix}^{-1} = \frac{1}{3} \begin{bmatrix}
		2 & 1 \\
		-1 & 1
		\end{bmatrix} = \begin{bmatrix}
		\frac{2}{3} & \frac{1}{3} \\
		- \frac{1}{3} & \frac{1}{3}
		\end{bmatrix}
	\]
	\[
		B^{-1} = \begin{bmatrix}
		2 & 1 \\
		1 & -2
		\end{bmatrix}^{-1} = -\frac{1}{5} \begin{bmatrix}
		-2 & -1 \\
		-1 & 2
		\end{bmatrix}^{-1} = \begin{bmatrix}
		\frac{2}{5} & \frac{1}{5} \\
		\frac{1}{5} & -\frac{2}{5}
		\end{bmatrix}
	\]
	\item
	\[
		AB = \begin{bmatrix}
		(1*2)+(-1*1) & (1*1)+(-1*-2) \\
		(1*2)+(2*1) & (1*1)+(2*-2)
		\end{bmatrix} = \begin{bmatrix}
		1 & 3 \\
		4 & -3
		\end{bmatrix}
	\]\\
	Again, $AB$ is inversible due to $(1*-3)-(3*4)=-15 \neq 0$
	\[
		(AB)^{-1} = \begin{bmatrix}
		1 & 3 \\
		4 & -3
		\end{bmatrix}^{-1} = \frac{1}{15} \begin{bmatrix}
		3 & 3 \\
		4 & -1
		\end{bmatrix} = \begin{bmatrix}
		\frac{1}{5} & \frac{1}{5} \\
		\frac{4}{15} & -\frac{1}{15}
		\end{bmatrix}
	\]
	\[
		B^{-1}A^{-1} = \begin{bmatrix}
		\frac{2}{5} & \frac{1}{5} \\
		\frac{1}{5} & -\frac{2}{5}
		\end{bmatrix}\begin{bmatrix}
		\frac{2}{3} & \frac{1}{3} \\
		- \frac{1}{3} & \frac{1}{3}
		\end{bmatrix} = \begin{bmatrix}
		(\frac{2}{5} * \frac{2}{3}) + (\frac{1}{5} * -\frac{1}{3}) & (\frac{2}{5} * \frac{1}{3}) + (\frac{1}{5} * \frac{1}{3}) \\
		(\frac{1}{5} * \frac{2}{3}) + (-\frac{2}{5} * -\frac{1}{3}) & (\frac{1}{5} * \frac{1}{3}) + (-\frac{2}{5} * \frac{1}{3})
		\end{bmatrix} = \begin{bmatrix}
		\frac{1}{5} & \frac{1}{5} \\
		\frac{4}{15} & -\frac{1}{15}
		\end{bmatrix}
	\]
	\[
		(AB)^{-1} = B^{-1}A^{-1} = \begin{bmatrix}
		\frac{1}{5} & \frac{1}{5} \\
		\frac{4}{15} & -\frac{1}{15}
		\end{bmatrix}
	\]
	\item
	For transposing a simple $2 * 2$ matrix it is only to swap $b$ and $c$. Thus giving us the matrices
	\[
		A^T = \begin{bmatrix}
		1 & 1 \\
		-1 & 2
		\end{bmatrix} \quad B^T = \begin{bmatrix}
		2 & 1 \\
		1 & -2
		\end{bmatrix} \quad (AB)^T = \begin{bmatrix}
		1 & 4 \\
		3 & -3
		\end{bmatrix}
	\]
	\[
		B^TA^T = \begin{bmatrix}
		1 & 1 \\
		-1 & 2
		\end{bmatrix} \begin{bmatrix}
		2 & 1 \\
		1 & -2
		\end{bmatrix} = \begin{bmatrix}
		2+-1 & 2+2 \\
		1+2 & 1+-4
		\end{bmatrix} = \begin{bmatrix}
		1 & 4 \\
		3 & -3
		\end{bmatrix}
	\]
	\[
		B^TA^T = (AB)^T = \begin{bmatrix}
		1 & 4 \\
		3 & -3
		\end{bmatrix}
	\]
	\item
	We know the solution to the eigenvalues can be calculated by $det(\lambda I-D)=0$. Therefore given D we can see
	\[
		D = \begin{bmatrix}
		1 & 0 & 0 \\
		1 & 2 & 0 \\
		2 & 0 & 3
		\end{bmatrix} \quad \lambda I_3 = \begin{bmatrix}
		\lambda & 0 & 0 \\
		0 & \lambda & 0 \\
		0 & 0 & \lambda
		\end{bmatrix}
	\]
	\[
		\lambda I_3 - D = \begin{bmatrix}
		\lambda - 1 & 0 & 0 \\
		-1 & \lambda - 2 & 0 \\
		-2 & 0 & \lambda - 3
		\end{bmatrix}
	\]\\
	
	Now given the matrix the determinant does get simple at both $b$ and $c$ is zero, we're left with the equation $(\lambda - 1)(\lambda - 2)(\lambda - 3)=0$ giving the possible solutions of $(\lambda = 1 \vee \lambda = 2 \vee \lambda = 3)$
	
	\item
	Given the matrix $D$ and the lambdas from above, we should be able to find an associated eigenvector to each of the corresponding values.\\
	
	\begin{minipage}[t]{0.3\textwidth}
	$D - {\lambda}_1 I$
	\[
		\begin{bmatrix}
		0 & 0 & 0 \\
		1 & 1 & 0 \\
		2 & 0 & 2
		\end{bmatrix}
	\]
	\[
		\begin{bmatrix}
		1 & -1 & 2 \\
		1 & 1 & 0 \\
		0 & 0 & 0
		\end{bmatrix}
	\]
		\[
		\begin{bmatrix}
		1 & 1 & 0 \\
		0 & 1 & -1 \\
		0 & 0 & 0
		\end{bmatrix}
	\]
	\end{minipage} \quad \vline \quad \begin{minipage}[t]{0.3\textwidth}
	$D - {\lambda}_2 I$
	\[
		\begin{bmatrix}
		-1 & 0 & 0 \\
		1 & 0 & 0 \\
		2 & 0 & 1
		\end{bmatrix}
	\]
	\[
		\begin{bmatrix}
		1 & 0 & 1 \\
		0 & 0 & 1 \\
		0 & 0 & 0
		\end{bmatrix}
	\]
	\end{minipage} \quad \vline \quad \begin{minipage}[t]{0.3\textwidth}
	$D - {\lambda}_3 I$	
	\[
		\begin{bmatrix}
		-2 & 0 & 0 \\
		1 & -1 & 0 \\
		2 & 0 & 0
		\end{bmatrix}
	\]
	\[
		\begin{bmatrix}
		1 & 0 & 0 \\
		0 & 1 & 0 \\
		0 & 0 & 0
		\end{bmatrix}
	\]
	\end{minipage}
	
	\begin{minipage}[t]{0.3\textwidth}	
	\[
		E_{\lambda = 1} = \begin{bmatrix}
		-1 \\ 1 \\ 1
		\end{bmatrix}
	\]
	\end{minipage} \quad \vline \quad \begin{minipage}[t]{0.3\textwidth}	
	\[
		E_{\lambda = 2} = \begin{bmatrix}
		0 \\ 1 \\ 0
		\end{bmatrix}
	\]
	\end{minipage} \quad \vline \quad \begin{minipage}[t]{0.3\textwidth}	
	\[
		E_{\lambda = 3} = \begin{bmatrix}
		0 \\ 0 \\ 1
		\end{bmatrix}
	\]
	\end{minipage}
	
    \end{enumerate}
    
    \newpage
    \mysubsection{Calculus}
    \begin{enumerate}
    \item
    Finding the partial derivatives we simply remove the "unknown" variable and find the derivative.
    
    \[
    		{\partial}_x f(x,y) = 2x-4 = 2(x-2)
    \]
    \[
    		{\partial}_y f(x,y) = 2y+6 = 2(y+3)
    \]
    
    \item
    By using the partial derivatives from above, we can find their zero value and use that to find the critical point.
    
    \begin{minipage}[t]{0.5\textwidth}
	${\partial}_x f(x,y)$    
    \[
    		2x-4 = 0
    \]
    \[
    		2x = 4
    \]
    \[
    		x = 2
    \]
    \end{minipage} \quad \vline \quad \begin{minipage}[t]{0.5\textwidth}
	${\partial}_y f(x,y)$    
    \[
    		2y+6 = 0
    \]
    \[
    		2y = -6
    \]
    \[
    		y = -3
    \]    
    \end{minipage}

	Which means we find the critical point located at $(2,-3)$ where $f(2,-3) = 2^2 + -3^2 - (4*2) + (6*-3) + 13 = -18$
    
    \item
	Factoring both partial derivatives (and forgetting the one constant) we are left with the new function $f(x,y) = 2(x-2)^2 + 2(y+3)^2$
	
    \end{enumerate}
    
    \mysubsection{Probabilities}
	}    
    
    
    % % % % % % % % % % % % % % % % %
    %
    %	ADD REFERANCES
    %
    %\bibliography{citation-db}
    %\addcontentsline{toc}{section}{References}

\end{document}